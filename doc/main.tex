\documentclass[letterpaper]{template} % a4paper for A4

\profilepic{avatar.jpg} % Profile picture

\cvname{Simone Romagnoli} % Your name
\cvjobtitle{Software Developer Intern} % Job title/career

\cvdate{} % Date of birth
\cvnumberphone{+39 347 241 0530} % Phone number
\cvmail{simone.romagnoli.21@gmail.com} % Email address
\cvsite{Via Corinaldo 80, Cesena (FC)}
\cvgithub{https://github.com/SimoneRomagnoli}{github.com/SimoneRomagnoli} % Personal website
\cvaddress{} % Personal website

\begin{document}

\profile{Sto finendo gli studi universitari per diventare ingegnere informatico. Vorrei sperimentare il ruolo di software developer in un contesto che mi permetta di applicare le mie conoscenze in team e mi aiuti a migliorare le mie abilità, con particolare attenzione ai principi di progettazione del software.}
\languages{{Spagnolo - B2/4}, {Inglese - C1/5}, {Italiano - Madrelingua/6}}
\skillstext{}


\hobby{
    \begin{itemize}[leftmargin=*]
        \item Pallacanestro: la mia passione per il basket mi ha permesso di sviluppare una forte attitudine per la \textit{leadership} e per il gioco di squadra.
        \item Editing Foto/Video: ho sviluppato esperienza con \textit{Adobe Premiere Pro} e \textit{Adobe Photoshop}.
        \item Giochi di logica: mi piace giocare a scacchi e risolvere il cubo di Rubik.
    \end{itemize}
}

\makeprofile

\section{Istruzione}

\begin{cvbox}
    \cvboxitem{2017}{Diploma di Liceo Scientifico}{Liceo Scientifico A. Righi - Cesena}{}
    \cvboxitem{2020}{Laurea Triennale in Ingegneria e Scienze Informatiche}{UniBo - Cesena}{Tesi in \textit{High Performance Computing}: Analisi del linguaggio x10 per architetture parallele: il caso di studio dell’algoritmo \textit{Gift Wrapping}.}
    \cvboxitem{in corso}{Laurea Magistrale in Ingegneria e Scienze Informatiche}{UniBo - Cesena}{Alcuni progetti che hanno contribuito all'apprendimento di competenze tecniche e trasversali:
    \begin{itemize}
        \item Sviluppo di un gioco \textit{strategy} in stile \textit{Tower Defense}: \href{https://github.com/SimoneRomagnoli/pps-popit}{Pop-It}.
        \item Sviluppo di un sistema per il riconoscimento di mascherine indossate e per il rispetto del distanziamento sociale: \href{https://github.com/alessandro-marcantoni/mask-detection-social-distancing-var}{Mask Detection \& Social Distancing}.
        \item Simulazione di un progetto con tecniche di \textit{project management}: \href{https://github.com/SmartWasteCollection/project-management/releases/tag/v1.0.0}{Smart Waste Collection}.
    \end{itemize}
    }
\end{cvbox}

\section{Esperienze Lavorative}

\begin{cvbox}
    \cvboxitem{Cameriere}{Lavoro Full-Time (luglio - agosto 2017)}{Hotel Imperiale - Cesenatico}{Regolare attività di cameriere da sala in hotel. Lavoro che ha insegnato a mantenere una costante attenzione e ad affrontare le difficoltà e le fatiche con sorriso e motivazione.}
    \cvboxitem{Tirocinante}{Tirocinio Curriculare (marzo - maggio 2020)}{BSD Software - Cesena}{Prima esperienza lavorativa in ambito Web; sviluppo di front end e back end di un applicativo gestionale di tirocini; approfondimento dei linguaggi \textit{Javascript}, \textit{JQuery}, \textit{C\#} e apprendimento del framework \textit{Knockout}.}
    \cvboxitem{Scaffalista}{Lavoro Part-Time Serale (luglio 2018 - giugno 2021)}{Conad SuperOtto - Cesena}{Caricamento veloce ed ordinato degli scaffali con eventuale pareggiamento e pulizia delle corsie. Capocantiere della sede di Cesena per la cooperativa Alimondo. Esperienza che insegna la gestione e la coordinazione del lavoro proprio e dei colleghi.}

\end{cvbox}

\section{Competenze}

\begin{itemize}
    \item \textbf{Programmazione in C} - abilità nel riconoscimento e nello sviluppo di strutture dati anche complesse.
    \item \textbf{Programmazione ad oggetti (\textit{Java}, \textit{Kotlin}, \textit{C\#})} - esperienza nella programmazione ad oggetti; conoscenza di diversi pattern di programmazione utili e forte rispetto dei principi di design del paradigma a oggetti.
    \item \textbf{Programmazione funzionale (\textit{Scala})} - abilità nella programmazione funzionale, soprattutto in \textit{Scala}.
    \item \textbf{Programmazione Web} - conoscenza profonda di \textit{HTTP}, \textit{CSS}, \textit{PHP} e \textit{Javascript} per lo sviluppo di applicazioni web; abilità di sviluppo con lo stack \textit{MEVN}.
    \item \textbf{Programmazione concorrente e distribuita} - abilità nello sviluppo di programmi di \textit{High Performance Computing} in linguaggi come \textit{OpenMP}, \textit{MPI}, \textit{CUDA} e \textit{X10}; attenzione sul design e sviluppo di sistemi a microservizi.
    \item \textbf{DevOps} - ampia conoscenza del \textit{DVCS Git} e esperienza nell'utilizzo di \textit{Git Flow}; esperienza nell'applicazione di flussi di \textit{continuous integration} e \textit{continuous delivery} soprattutto tramite le \textit{GitHub Actions}; conoscenza di meccanismi di deployment con \textit{Docker}.
    \item \textbf{Competenze aggiuntive} - abilità nel redigere documenti \textit{LaTex}; conoscenza della programmazione logica; ampia conoscenza dei sistemi operativi \textit{Windows} e \textit{Linux}; dimestichezza con ambiente e terminale \textit{Linux} (preferito).
\end{itemize}

\vspace{6mm}

\scriptsize{Autorizzo il trattamento dei miei dati personali presenti nel curriculum vitae ai sensi del Decreto Legislativo 30 giugno 2003, n. 196 e del GDPR (Regolamento UE 2016/679).}

\end{document} 
