\documentclass[letterpaper]{template} % a4paper for A4

\profilepic{avatar.jpg} % Profile picture

\cvname{Simone Romagnoli} % Your name
\cvjobtitle{Software Developer Intern} % Job title/career

\cvdate{} % Date of birth
\cvnumberphone{+39 347 241 0530} % Phone number
\cvmail{simone.romagnoli.21@gmail.com} % Email address
\cvsite{Via Corinaldo 80, Cesena (FC)}
\cvgithub{https://github.com/SimoneRomagnoli}{github.com/SimoneRomagnoli} % Personal website
\cvaddress{} % Personal website

\begin{document}

\profile{Sto finendo gli studi universitari per diventare ingegnere informatico. Vorrei sperimentare il ruolo di software developer in un contesto che mi permetta di applicare le mie conoscenze in team e mi aiuti a migliorare le mie abilità, con particolare attenzione ai principi di progettazione del software.}
\languages{{Spagnolo - B2/4}, {Inglese - C1/5}, {Italiano - Madrelingua/6}}
\skillstext{}


\hobby{
    \begin{itemize}[leftmargin=*]
        \item Pallacanestro: la mia passione per il basket mi ha permesso di sviluppare una forte attitudine per la \textit{leadership} e per il gioco di squadra.
        \item Editing Foto/Video: ho sviluppato esperienza con \textit{Adobe Premiere Pro} e \textit{Adobe Photoshop}.
        \item Giochi di logica: mi piace giocare a scacchi e risolvere il cubo di Rubik.
    \end{itemize}
}

\makeprofile

\section{Istruzione}

\begin{cvbox} % Environment for a list with descriptions
    \cvboxitem{in corso}{Laurea Magistrale in Ingegneria e Scienze Informatiche}{UniBo - Cesena}{Specializzazione in ambito ingegneristico.}
    \cvboxitem{2020}{Laurea Triennale in Ingegneria e Scienze Informatiche}{UniBo - Cesena}{I progetti che hanno contribuito maggiormente all'apprendimento di competenze tecniche e trasversali sono stati:
        \begin{itemize}
            \item \textbf{Big Hunt} - riproduzione del gioco sparatutto "light gun" Duck Hunt (1984) sviluppato in Java;
            \item \textbf{WhiteBall} - gioco per smartphone (Android) sviluppato in Java;
            \item \textbf{FOOL Compiler} - sviluppo di un compilatore di un linguaggio funzionale e object-oriented sviluppato in Java;
            \item \textbf{Parallel Convex Hull} - progetto di tesi universitaria triennale, parallelizzazione dell'algoritmo Gift Wrapping che trova l'inviluppo convesso di un insieme di punti, sviluppato in X10;
            \item \textbf{e20} - applicazione web (senza l'utilizzo di framework) che simula un sito di eventi (in stile TicketOne).
        \end{itemize}
    }
    \cvboxitem{2017}{Diploma di Liceo Scientifico A. Righi}{Cesena}{Studio regolare e con voti alti con attenzione all'apprendimento veloce con un buon metodo di studio.}
\end{cvbox}

\section{Esperienze}

\begin{cvbox} % Environment for a list with descriptions
    \cvboxitem{Scaffalista}{Lavoro Part-Time Serale (luglio 2018 - oggi)}{Conad SuperOtto - Cesena}{Caricamento veloce ed ordinato degli scaffali con eventuale pareggiamento e pulizia delle corsie. Capocantiere della sede di Cesena per la cooperativa Alimondo. Esperienza che insegna la gestione e la coordinazione del lavoro proprio e dei colleghi.}
    \cvboxitem{Tirocinante}{Tirocinio Curriculare (marzo - maggio 2020)}{BSD Software - Cesena}{Prima esperienza lavorativa in ambito Web; sviluppo di front end e back end di un applicativo gestionale di tirocini; approfondimento dei linguaggi Javascript, JQuery, C\# e apprendimento del framework Knockout.}
    \cvboxitem{Cameriere}{Lavoro Full-Time (luglio - agosto 2017)}{Hotel Imperiale - Valverde}{Regolare attività di cameriere da sala in hotel. Lavoro che ha insegnato a mantenere una costante attenzione e ad affrontare le difficoltà e le fatiche con sorriso e motivazione.}
\end{cvbox}

\section{Competenze}

\begin{itemize}
    \item \textbf{Programmazione in C/C\#} - abilità nel riconoscimento e nello sviluppo di strutture dati anche complesse; esperienza globale, specialmente nel lato back end per lo sviluppo di applicazioni;
    \item \textbf{Programmazione ad oggetti (Java)} - esperienza nella programmazione ad oggetti, soprattutto con Java; conoscenza di alcuni pattern di programmazione utili e forte rispetto dei principi di design del paradigma a oggetti;
    \item \textbf{Programmazione funzionale (Scala)} - nozioni principali sulla programmazione funzionale in Scala (abilità in sviluppo nel corso dei miei studi magistrali);
    \item \textbf{Programmazione Web} - conoscenza profonda dei linguaggi HTTP, CSS, PHP e Javascript per lo sviluppo di applicazioni web; nozioni sullo stack MEAN (abilità in sviluppo nel corso dei miei studi magistrali);
    \item \textbf{Programmazione concorrente e distribuita} - abilità nello sviluppo di programmi di High Performance Computing in linguaggi come OpenMP, MPI, CUDA e X10; attenzione sul design di sistemi distribuiti ad agenti (abilità in sviluppo nel corso dei miei studi magistrali);
     \item \textbf{Competenze aggiuntive} - ampia conoscenza del DVCS Git; dimestichezza con ambiente e terminale Linux; abilità nel redigere documenti LaTex.
\end{itemize}

\vspace{6mm}

\scriptsize{Autorizzo il trattamento dei miei dati personali presenti nel curriculum vitae ai sensi del Decreto Legislativo 30 giugno 2003, n. 196 e del GDPR (Regolamento UE 2016/679).}

\end{document} 
